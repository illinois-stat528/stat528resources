% Options for packages loaded elsewhere
\PassOptionsToPackage{unicode}{hyperref}
\PassOptionsToPackage{hyphens}{url}
\PassOptionsToPackage{dvipsnames,svgnames,x11names}{xcolor}
%
\documentclass[
  ignorenonframetext,
]{beamer}
\usepackage{pgfpages}
\setbeamertemplate{caption}[numbered]
\setbeamertemplate{caption label separator}{: }
\setbeamercolor{caption name}{fg=normal text.fg}
\beamertemplatenavigationsymbolsempty
% Prevent slide breaks in the middle of a paragraph
\widowpenalties 1 10000
\raggedbottom
\setbeamertemplate{part page}{
  \centering
  \begin{beamercolorbox}[sep=16pt,center]{part title}
    \usebeamerfont{part title}\insertpart\par
  \end{beamercolorbox}
}
\setbeamertemplate{section page}{
  \centering
  \begin{beamercolorbox}[sep=12pt,center]{part title}
    \usebeamerfont{section title}\insertsection\par
  \end{beamercolorbox}
}
\setbeamertemplate{subsection page}{
  \centering
  \begin{beamercolorbox}[sep=8pt,center]{part title}
    \usebeamerfont{subsection title}\insertsubsection\par
  \end{beamercolorbox}
}
\AtBeginPart{
  \frame{\partpage}
}
\AtBeginSection{
  \ifbibliography
  \else
    \frame{\sectionpage}
  \fi
}
\AtBeginSubsection{
  \frame{\subsectionpage}
}
\usepackage{amsmath,amssymb}
\usepackage{lmodern}
\usepackage{iftex}
\ifPDFTeX
  \usepackage[T1]{fontenc}
  \usepackage[utf8]{inputenc}
  \usepackage{textcomp} % provide euro and other symbols
\else % if luatex or xetex
  \usepackage{unicode-math}
  \defaultfontfeatures{Scale=MatchLowercase}
  \defaultfontfeatures[\rmfamily]{Ligatures=TeX,Scale=1}
\fi
% Use upquote if available, for straight quotes in verbatim environments
\IfFileExists{upquote.sty}{\usepackage{upquote}}{}
\IfFileExists{microtype.sty}{% use microtype if available
  \usepackage[]{microtype}
  \UseMicrotypeSet[protrusion]{basicmath} % disable protrusion for tt fonts
}{}
\makeatletter
\@ifundefined{KOMAClassName}{% if non-KOMA class
  \IfFileExists{parskip.sty}{%
    \usepackage{parskip}
  }{% else
    \setlength{\parindent}{0pt}
    \setlength{\parskip}{6pt plus 2pt minus 1pt}}
}{% if KOMA class
  \KOMAoptions{parskip=half}}
\makeatother
\usepackage{xcolor}
\newif\ifbibliography
\setlength{\emergencystretch}{3em} % prevent overfull lines
\providecommand{\tightlist}{%
  \setlength{\itemsep}{0pt}\setlength{\parskip}{0pt}}
\setcounter{secnumdepth}{-\maxdimen} % remove section numbering
\usepackage{graphicx}
\usepackage{bm}
\usepackage{array}
\usepackage{amsmath}
\usepackage{amsthm}
\usepackage{amsfonts}
\usepackage{amssymb}
\usepackage{tikz-cd}
\usepackage{url}
\definecolor{foreground}{RGB}{255,255,255}
\definecolor{background}{RGB}{34,28,54}
\definecolor{title}{RGB}{105,165,255}
\definecolor{gray}{RGB}{175,175,175}
\definecolor{lightgray}{RGB}{225,225,225}
\definecolor{subtitle}{RGB}{232,234,255}
\definecolor{hilight}{RGB}{112,224,255}
\definecolor{vhilight}{RGB}{255,111,207}
\setbeamertemplate{footline}[page number]
\ifLuaTeX
  \usepackage{selnolig}  % disable illegal ligatures
\fi
\IfFileExists{bookmark.sty}{\usepackage{bookmark}}{\usepackage{hyperref}}
\IfFileExists{xurl.sty}{\usepackage{xurl}}{} % add URL line breaks if available
\urlstyle{same} % disable monospaced font for URLs
\hypersetup{
  pdftitle={STAT 528 - Advanced Regression Analysis II},
  pdfauthor={Count response regression (part I)},
  colorlinks=true,
  linkcolor={Maroon},
  filecolor={Maroon},
  citecolor={Blue},
  urlcolor={blue},
  pdfcreator={LaTeX via pandoc}}

\title{STAT 528 - Advanced Regression Analysis II}
\author{Count response regression (part I)}
\date{}
\institute{Daniel J. Eck\\
Department of Statistics\\
University of Illinois}

\begin{document}
\frame{\titlepage}

\begin{frame}
\newcommand{\R}{\mathbb{R}}
\newcommand{\Prob}{\mathbb{P}}
\newcommand{\Proj}{\textbf{P}}
\newcommand{\Hcal}{\mathcal{H}}
\newcommand{\rootn}{\sqrt{n}}
\newcommand{\p}{\mathbf{p}}
\newcommand{\E}{\text{E}}
\newcommand{\Var}{\text{Var}}
\newcommand{\Cov}{\text{Cov}}
\newcommand{\mubf}{\bm{\mu}}
\newcommand{\logit}{\text{logit}}

\newtheorem{cor}{Corollary}
\newtheorem{lem}{Lemma}
\newtheorem{thm}{Theorem}
\newtheorem{defn}{Definition}
\newtheorem{prop}{Proposition}
\end{frame}

\begin{frame}{Last time}
\protect\hypertarget{last-time}{}
\begin{itemize}
\tightlist
\item
  basic diagnostics for binary response models
\item
  probit regression and threshold modeling
\item
  basic causal inference
\end{itemize}
\end{frame}

\begin{frame}{Learning Objectives Today}
\protect\hypertarget{learning-objectives-today}{}
\begin{itemize}
\tightlist
\item
  Poisson regression
\end{itemize}
\end{frame}

\begin{frame}{Background}
\protect\hypertarget{background}{}
We suppose that we have a sample of data \((y_i,x_i)\),
\(i = 1,\ldots, n\) where

\begin{itemize}
\tightlist
\item
  \(y_i\) is a scalar response variable
\item
  \(x_i\) is a vector of predictors.
\end{itemize}

Recall from the exponential family notes that the log likelihood of the
exponential family is of the form \begin{equation} \label{expolog}
    l(\theta) = \langle y, \theta \rangle - c(\theta),
\end{equation} where

\begin{itemize}
\tightlist
\item
  \(y \in \mathbb{R}^n\) is a vector statistic having components
\item
  \(\theta \in \mathbb{R}^n\) is the canonical parameter vector.
\end{itemize}

In those notes \(\theta\) is unconstrained and the likelihood
\eqref{expolog} corresponds to a saturated regression model, one
parameter for every observation.
\end{frame}

\begin{frame}{}
\protect\hypertarget{section}{}
A canonical linear submodel of an exponential family is a submodel
having parameterization \[
  \theta = M\beta,
\] and log likelihood \begin{equation} \label{subloglike}
  l(\beta) = \langle M'y, \beta \rangle - c(M\beta).
\end{equation}

In an exponential family GLM, the saturated model canonical parameter
vector \(\theta\) is ``linked'\,' to the saturated model mean value
parameter vector through the change-of-parameter mappings \(g(\theta)\).

\vspace{12pt}

We can write \[
 \mu = \text{E}_\theta(Y) = g(M\beta) 
\] which implies that we can write \[
  g^{-1}\left(\text{E}_\theta(Y)\right) = M\beta.
\]
\end{frame}

\begin{frame}{Poisson regression model}
\protect\hypertarget{poisson-regression-model}{}
The Poisson regression model {[}and its variants{]} is one of the more
widely used and studied exponential family GLMs in practice.

\vspace{12pt}

The Poisson regression model is used for analyzing a count response
variable, \(y_i \in \{0,1,2,3,\ldots\}\).

\vspace{12pt}

The Poisson regression model allows for users to model the rate as a
function of covariates.
\end{frame}

\begin{frame}{}
\protect\hypertarget{section-1}{}
For a count response variable \(Y\) and a vector of predictors \(X\),
let \(\mu(x) = \text{E}(Y|X = x)\). The Poisson regression model is then
\begin{equation} \label{loglink}
  \mu(x) = \text{E}(Y|X = x) = \exp\left(x'\beta\right).
\end{equation} Equivalently, \[
  \log\left(\mu(x)\right) = x'\beta.
\] In vector notation, we can express the above as \[
  \bm{\mu}= \exp(M\beta) \quad \text{and} \quad \text{log}(\bm{\mu}) = M\beta
\] where the above \(\exp(\cdot)\) and log\((\cdot)\) operations are
understood as componentwise operations.
\end{frame}

\begin{frame}{}
\protect\hypertarget{section-2}{}
Let's consider the log likelihood of a sample of independent Poisson
random variables \begin{align*}
  \sum_{i=1}^n y_i\log(\mu_i) - \mu_i 
    &=  \sum_{i=1}^n y_i\theta_i - \exp(\theta_i)
\end{align*} where \[
  \theta_i = \log\left(\mu_i\right) \qquad \text{and} \qquad \mu_i = \exp(\theta_i) = g(\theta_i).
\]

We see that the Poisson regression model with log link is the same as
the canonical linear submodel of an exponential family.

\vspace{12pt}

The link function \(g^{-1}\) is the logarithmic function. Hence the name
log-linear models.
\end{frame}

\end{document}
