% Options for packages loaded elsewhere
\PassOptionsToPackage{unicode}{hyperref}
\PassOptionsToPackage{hyphens}{url}
\PassOptionsToPackage{dvipsnames,svgnames,x11names}{xcolor}
%
\documentclass[
  ignorenonframetext,
]{beamer}
\usepackage{pgfpages}
\setbeamertemplate{caption}[numbered]
\setbeamertemplate{caption label separator}{: }
\setbeamercolor{caption name}{fg=normal text.fg}
\beamertemplatenavigationsymbolsempty
% Prevent slide breaks in the middle of a paragraph
\widowpenalties 1 10000
\raggedbottom
\setbeamertemplate{part page}{
  \centering
  \begin{beamercolorbox}[sep=16pt,center]{part title}
    \usebeamerfont{part title}\insertpart\par
  \end{beamercolorbox}
}
\setbeamertemplate{section page}{
  \centering
  \begin{beamercolorbox}[sep=12pt,center]{part title}
    \usebeamerfont{section title}\insertsection\par
  \end{beamercolorbox}
}
\setbeamertemplate{subsection page}{
  \centering
  \begin{beamercolorbox}[sep=8pt,center]{part title}
    \usebeamerfont{subsection title}\insertsubsection\par
  \end{beamercolorbox}
}
\AtBeginPart{
  \frame{\partpage}
}
\AtBeginSection{
  \ifbibliography
  \else
    \frame{\sectionpage}
  \fi
}
\AtBeginSubsection{
  \frame{\subsectionpage}
}
\usepackage{amsmath,amssymb}
\usepackage{lmodern}
\usepackage{iftex}
\ifPDFTeX
  \usepackage[T1]{fontenc}
  \usepackage[utf8]{inputenc}
  \usepackage{textcomp} % provide euro and other symbols
\else % if luatex or xetex
  \usepackage{unicode-math}
  \defaultfontfeatures{Scale=MatchLowercase}
  \defaultfontfeatures[\rmfamily]{Ligatures=TeX,Scale=1}
\fi
% Use upquote if available, for straight quotes in verbatim environments
\IfFileExists{upquote.sty}{\usepackage{upquote}}{}
\IfFileExists{microtype.sty}{% use microtype if available
  \usepackage[]{microtype}
  \UseMicrotypeSet[protrusion]{basicmath} % disable protrusion for tt fonts
}{}
\makeatletter
\@ifundefined{KOMAClassName}{% if non-KOMA class
  \IfFileExists{parskip.sty}{%
    \usepackage{parskip}
  }{% else
    \setlength{\parindent}{0pt}
    \setlength{\parskip}{6pt plus 2pt minus 1pt}}
}{% if KOMA class
  \KOMAoptions{parskip=half}}
\makeatother
\usepackage{xcolor}
\newif\ifbibliography
\setlength{\emergencystretch}{3em} % prevent overfull lines
\providecommand{\tightlist}{%
  \setlength{\itemsep}{0pt}\setlength{\parskip}{0pt}}
\setcounter{secnumdepth}{-\maxdimen} % remove section numbering
\usepackage{graphicx}
\usepackage{bm}
\usepackage{array}
\usepackage{amsmath}
\usepackage{amsthm}
\usepackage{amsfonts}
\usepackage{amssymb}
\usepackage{tikz-cd}
\usepackage{url}
\definecolor{foreground}{RGB}{255,255,255}
\definecolor{background}{RGB}{34,28,54}
\definecolor{title}{RGB}{105,165,255}
\definecolor{gray}{RGB}{175,175,175}
\definecolor{lightgray}{RGB}{225,225,225}
\definecolor{subtitle}{RGB}{232,234,255}
\definecolor{hilight}{RGB}{112,224,255}
\definecolor{vhilight}{RGB}{255,111,207}
\setbeamertemplate{footline}[page number]
\ifLuaTeX
  \usepackage{selnolig}  % disable illegal ligatures
\fi
\IfFileExists{bookmark.sty}{\usepackage{bookmark}}{\usepackage{hyperref}}
\IfFileExists{xurl.sty}{\usepackage{xurl}}{} % add URL line breaks if available
\urlstyle{same} % disable monospaced font for URLs
\hypersetup{
  pdftitle={STAT 528 - Advanced Regression Analysis II},
  pdfauthor={Exponential family theory},
  colorlinks=true,
  linkcolor={Maroon},
  filecolor={Maroon},
  citecolor={Blue},
  urlcolor={blue},
  pdfcreator={LaTeX via pandoc}}

\title{STAT 528 - Advanced Regression Analysis II}
\author{Exponential family theory}
\date{}
\institute{Daniel J. Eck\\
Department of Statistics\\
University of Illinois}

\begin{document}
\frame{\titlepage}

\begin{frame}
\newcommand{\R}{\mathbb{R}}
\newcommand{\Prob}{\mathbb{P}}
\newcommand{\Proj}{\textbf{P}}
\newcommand{\Hcal}{\mathcal{H}}
\newcommand{\rootn}{\sqrt{n}}
\newcommand{\p}{\mathbf{p}}
\newcommand{\E}{\text{E}}
\newcommand{\Var}{\text{Var}}
\newcommand{\Cov}{\text{Cov}}
\end{frame}

\begin{frame}{Agenda for today}
\protect\hypertarget{agenda-for-today}{}
\begin{itemize}
\tightlist
\item
  Course software and GitHub
\item
  Go over basics of exponential family theory
\end{itemize}
\end{frame}

\begin{frame}{Exponential family}
\protect\hypertarget{exponential-family}{}
An \emph{exponential family of distributions} is a parametric
statistical model having log likelihood that takes the form
\begin{equation} \label{expolog}
    l(\theta) = \langle y,\theta \rangle - c(\theta),
\end{equation} where \(y\) is a vector statistic and \(\theta\) is a
vector parameter, and

\begin{itemize}
\tightlist
\item
  \(\langle y,\theta \rangle\) is the usual inner product,
\item
  \(c(\theta)\) is the cumulant function.
\end{itemize}

This uses the convention that terms that do not contain the parameter
vector can be dropped from a log likelihood; otherwise additional terms
also appear in \eqref{expolog}.

When the log likelihood can be expressed as \eqref{expolog} we say that
\(y\) is the \textbf{canonical statistic} and \(\theta\) is the
\textbf{canonical parameter}.
\end{frame}

\begin{frame}{Example: Binomial distribution}
\protect\hypertarget{example-binomial-distribution}{}
Let \(X \sim\) Binomial(\(n\),\(p\)) where \(0 < p < 1\). We can write
the log probability mass function for \(X\) \begin{align*}
  l(p) &= \log\left({n \choose x}\right) +  x\log(p) + (n-x)\log(1-p) \\
       &\propto \log\left({n \choose x}\right) +  x\log(p) + (n-x)\log(1-p) 
\end{align*} in exponential family form \[
  l(\theta) = \langle y,\theta \rangle - c(\theta).
\]
\end{frame}

\begin{frame}{Densities}
\protect\hypertarget{densities}{}
Let \(w\) represent the full data, then the densities have the form
\begin{equation} \label{expodens}
  f_\theta(w) = h(w)\exp\left(\langle Y(w),\theta\rangle - c(\theta)\right)
\end{equation} and the word ``density'\,' here can refer to a PMF, PDF,
or to a density with respect to a positive measure.

The \(h(w)\) arises from any term not containing the parameter that is
dropped in going from log densities to log likelihood as we saw on the
previous slide.

The function \(h\) has to be nonnegative, and any point \(w\) such that
\(h(w) = 0\) is not in the support of any distribution in the family.
\end{frame}

\begin{frame}{Example: Binomial distribution}
\protect\hypertarget{example-binomial-distribution-1}{}
Let \(X \sim\) Binomial(\(n\),\(p\)) where \(0 < p < 1\). We can write
the probability mass function for \(X\) \[
  f_p(x) = {n \choose x}p^x(1-p)^{n-x}
\] as an exponential family density \[
 f_{\theta}(w) = h(w)\exp\left( \langle Y(w),\theta \rangle - c(\theta)\right).
\]
\end{frame}

\begin{frame}{Example: Normal distribution}
\protect\hypertarget{example-normal-distribution}{}
Let \(W \sim N(\mu, \sigma^2)\). Then we can write \[
    f_{\mu,\sigma^2}(w) = \frac{1}{\sqrt{2\pi\sigma^2}}\exp\left(-\frac{(w-\mu)^2}{2\sigma^2}\right) 
\] as an exponential family density \[
 f_{\theta}(w) = h(w)\exp\left( \langle Y(w),\theta \rangle - c(\theta)\right),
\] where \[
  c(\theta) = \frac{1}{2}\left(\frac{\theta_1^2}{2\theta_2} - \log(2\theta_2)\right).
\]
\end{frame}

\begin{frame}{Cumulant functions}
\protect\hypertarget{cumulant-functions}{}
Being a density, \eqref{expodens} must sum, integrate, or sum-integrate
to one. Hence, \begin{align*}
    1 &= \int f_\theta(w) dw \\ 
      &= \int \exp\left(\langle Y(w),\theta\rangle - c(\theta)\right) h(w)dw \\
      &= \exp\left(-c(\theta)\right) \int \exp\left(\langle Y(w),\theta\rangle\right) h(w) dw.
\end{align*} Rearranging the above implies that \[
  c(\theta) = \log\left(\int \exp\left(\langle Y(w),\theta\rangle\right) h(w) dw\right).
\]
\end{frame}

\begin{frame}{}
\protect\hypertarget{section}{}
The cumulant function is the log Laplace transformation corresponding to
the \emph{generating measure} given by \[
  \lambda(dw) = h(w)dw
\] when the random variable is continuous. Under this formulation \[
  c(\theta) = \log\left(\int \exp\left(\langle Y(w),\theta\rangle\right) \lambda(dw)\right).
\]

In our log likelihood based definition of the exponential family
\eqref{expolog}, the dropped terms which do not appear in the log
likelihood are incorporated into the counting measure (discrete
distributions) or Lebesgue measure (continuous distributions).
\end{frame}

\end{document}
