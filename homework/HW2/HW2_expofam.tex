\documentclass[12pt]{article}
\usepackage[utf8]{inputenc}

\usepackage{amsmath}
\usepackage{amsthm}
\usepackage{amsfonts}
%\usepackage{amscd}
\usepackage{amssymb}
\usepackage{graphicx}
\usepackage{mathtools}
\usepackage{natbib}
\usepackage{url}
\usepackage{graphicx,times}
\usepackage{tikz-cd}
\usepackage{array,epsfig,fancyheadings,rotating}
\usepackage{geometry}
\usepackage[usenames]{color}

\geometry{margin=1in}
\renewcommand{\baselinestretch}{1.2}

\newcommand{\R}{\mathbb{R}}
\newcommand{\Prob}{\mathbb{P}}
\newcommand{\Proj}{\textbf{P}}
\newcommand{\E}{\mathrm{E}}
\newcommand{\Hcal}{\mathcal{H}}
\newcommand{\rootn}{\sqrt{n}}

\newcommand{\norm}[1]{\left\lVert#1\right\rVert}
\newcommand{\indep}{\perp\!\!\!\perp}
\newcommand{\inner}[1]{\langle #1 \rangle}
\newcommand{\set}[1]{\{\, #1 \,\}}

\DeclareMathOperator{\E}{E}
\DeclareMathOperator{\tr}{tr}
\DeclareMathOperator{\etr}{etr}
\DeclareMathOperator{\Var}{Var}
\DeclareMathOperator{\MSE}{MSE}
\DeclareMathOperator{\vecop}{vec}
\DeclareMathOperator{\vech}{vech}


\newtheorem{cor}{Corollary}
\newtheorem{lem}{Lemma}
\newtheorem{thm}{Theorem}
\newtheorem{defn}{Definition}
\newtheorem{prop}{Proposition}


\pagestyle{fancy}
\def\n{\noindent}
\lhead[\fancyplain{} \leftmark]{}
\chead[]{}
\rhead[]{\fancyplain{}\rightmark}
\cfoot{}
%\headrulewidth=0pt


\setcounter{page}{1}
\setcounter{equation}{0}

\newcommand\red[1]{{\color{red}#1}}

\allowdisplaybreaks

\title{Homework 2: Exponential Families}
\author{your name}
\date{Due: February 3rd at 11:59 PM} 



\begin{document}

\maketitle


\noindent{\bf Problem 1}: Verify that displayed equation 7 in the exponential family notes holds for the binomial distribution, the Poisson distribution, and the normal distribution with both $\mu$ and $\sigma^2$ unknown.

\vspace*{1cm}

\noindent{\bf Problem 2}: Show that the second derivative of the map $h$ (displayed equation 11 in the exponential family notes) is equal to $-\nabla^2 c(\theta)$ and justify that this matrix is negative definite when the exponential family model is identifiable.

\vspace*{1cm}

\noindent{\bf Problem 3}: The above problem is one of the steps needed to finish the proof of Theorem 2 in the exponential family notes. Finish the proof of Theorem 2.

\vspace*{1cm}

\noindent{\bf Problem 4}: Let $Y$ be a regular full exponential family with canonical parameter vector $\theta$. Verify that $Y$ is sub-exponential.

\vspace*{1cm}

\noindent{\bf Problem 5}: In the notes it was claimed that the negative and/or a scalar products of $\sum_{i=1}^n\{y_i - \nabla c(\theta)\}$ are also sub-exponential (page 15). Show that this is in fact true when the observations $y_i$ are iid from a regular full exponential family.

\vspace*{1cm}

\noindent{\bf Problem 6}: Derive the MLEs of the canonical parameters of the binomial distribution, the Poisson distribution, and the normal distribution with both $\mu$ and $\sigma^2$ unknown.

\vspace*{1cm}

\noindent{\bf Problem 7}: Derive the asymptotic distribution for the MLE of the submodel mean value parameter vector $\hat\tau$.

\vspace*{1cm}

\noindent{\bf Problem 8}: Prove Lemma 1 in the exponential family notes.

\vspace*{1cm}



\end{document}



